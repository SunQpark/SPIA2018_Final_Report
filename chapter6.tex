\chapter{Conclusion}\label{Ch:Conclusion}

By far, we have introduced our works done and examples of results. The result of output images are not as satisfying but considering the quantity and quality of sketch images collected from google image search, this amount of dependency on input quality might be considered acceptible. Building more generally appliable model is considered possible but may require more time and resources. In our opinion, there are still several schemes remaining that deserve experiments.

We experimented utilizing the `grouped convolution' to improve U-net structure and got results which are considered better by us. However, the absent of objective method for evaluation made it impossible to properly compare our models and choose better ones. So, getting proper measure for the performance of generated image would be most immediate goal for the case when this work is continued. After that, we expect to evaluate the value of ideas applied on our model and properly select better model, which would make possible to get images with much better quality.


% fill in what you learned here
Although our work, done as SPIA 2018 program, did not finished with so called `state of the art' result, as undergraduate students interested in deep learning and current state of technology, we expect this experience would someday help us to push the boundary of the mankind knowledge. 

\section{Acknowledgements}

Our team would like to acknowledge Professor Yu-Wing TAI and Dr. Ningchen YING for helpful discussion. Professor Yu-Wing TAI offered a lots of helpful suggestions to model development such as implementing U-net as well as suggestion to data collection. Dr Ningchen YING offered a general introduction and outlining to the project and his patience to guide the project. We would also like to thank Professor Shingyu Leung, Professor Avery CHING, HKUST and SNU MATH department for providing computational resources and a sight seeing trip to Macau for stimulating our creativity. 

\endinput
