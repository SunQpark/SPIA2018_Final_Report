% don't display the page number
%\thispagestyle{empty}

In this project, we investigated sketch-to-image translation by implementing CycleGAN to learn the mapping between human face sketch to realistic photograph. We used U-net to form the Generative Adversarial Networks This makes the model possible to train end to end from very few images and guaranteeing the performance of the mapping. %wait until finished the result section.
%This sample report serves two purposes.
%First, it introduces the RIPS ``house style''---preferences for how copy is set and laid out on a page.%%
%\footnote{
%R. M. Ritter, {\em New Hart's Rules:  The Handbook of Style for Writers and Editors}, Oxford University Press, 2005.
%}
%Second, by comparing this document with the {\LaTeX} source, it illustrates the effects of {\LaTeX} code on the resulting typeset.%
%\footnote{
%Location of the source code is provided in Appendix \ref{App:SourceLocation}.
%}

%\vspace{24pt}
%(The Abstract should succinctly summarize the purpose and results of the RIPS project. 
%Usually, it will be one paragraph of no more than half a page to one page in length.
%The Abstract is often the last major component to be written, since it is almost impossible to know what to say until you have essentially completed the project.

%The Abstract is self-contained.
%For example, unfamiliar acronyms should be used sparingly, and if used, should also be spelled out.
%References to the literature should be specified completely, not cited for look-up in the Bibliography.)%%
%\footnote{
%Note that in front matter the footnote reference can be a symbol, but in the body it is usually a number.
