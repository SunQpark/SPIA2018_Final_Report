\chapter{Dealing with bibliographic references}\label{Ch:References}


{\LaTeX} uses a tool called {\BibTeX} to help you manage bibliographic references.
The references are maintained in a separate file, with a \texttt{.bib}
extension. For example, in this sample RIPS report the references
are in the \texttt{Biblio.bib} file, which is then included
from the main document.

{\BibTeX} affords a variety of \emph{entry types} (also known as \emph{record types}) for bibliographic records of books, articles, proceedings, theses, and many others.  
A good discussion of these and of the \emph{field entries} used for each type, find the \textsc{Wikipedia} article for {\BibTeX}.
You can use any editor to create your own bibliographic records, but you might find useful open-source tools available on the web.%%
\footnote{
Appendix \ref{App:BibTeX-Records} lists some examples of record types and fields used for {\BibTeX} entries.
}
%%

If you would like to include a reference to a book or an article in the
{\BibTeX} file, you can either create its {\BibTeX} entry by hand, or get it 
ready-made as a BibTeX entry from a web source. 
\emph{Google Scholar}, which you can find on the internet, is free and easy to use after you have set up preferences to include citations in a {\BibTeX} format.
Ask your Academic Mentor for other suggestions.%%


After you update your report's references file, to see the updated
references displayed in the report you need to run {\LaTeX} on the
report, then run {\BibTeX} (on the report, not on the references
file), and then run {\LaTeX} on the report again, once or a couple
of times.%%
\footnote{
Some typesetters can operate in a mode that performs all these operations in the correct sequence automatically.
}
%%
Don't be surprised if it takes you awhile to get used to
this routine---you won't be the first. Look for the {\BibTeX} icon or \emph{Bibliography} tab
somewhere in your editor's toolbar.

For example, to cite the book \emph{More Math Into {\LaTeX}} by Gr\"{a}tzer \cite{gratzer}, first look in the \texttt{Biblio.bib} file to find that the entry for that reference is labeled \texttt{gratzer}.
Then enter the code \verb1\cite{gratzer}1 in your text.
After
\begin{itemize}
\item running {\LaTeX} on your document once,
\item then running {\BibTeX} on your document once,
\item then running {\LaTeX} on your document twice,
\end{itemize}
you will find the citation ``\cite{gratzer}'' obtained in the preceding paragraph, which is the correct number the reference for Gr\"{a}tzer's book winds up as in the Bibliography of this report.
Note that you do not run {\LaTeX} on the \texttt{.bib} file.

{\BibTeX} does take getting used to.
You may need some help from an experienced {\LaTeX}er when getting started.

\endinput
