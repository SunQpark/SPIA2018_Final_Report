\chapter{Background}\label{Ch:Background}
\section{Related Models}
There are two main approaches for image generation tasks: generative adversarial nets and variational autoencoder. In this project, only GAN-related methods will be considered and inplemented.
The original GAN proposed by Goodfellow can train a generator to learn the real image distribution by training a discriminator together, which can tell the difference between real images and fake images.

In the field of image translation trained with unpaired image datasets, 3 models, CycleGAN, DiscoGAN, DualGAN are doing the same thing
from high level perspective. They train two generative models to learn the mapping between two image domains by training two discriminators
at each domain together and considering the cycle-consistency loss, which ensures ignorable difference between images and reconstructed images.
\\
In detail, the differences are
CycleGAN uses instance normalization, patchGAN discriminator. To stabilize the training, use least square GAN. Replay buffer
              no random input z no drop out; L1 distance as cycle consistency

DualGAN use generator and discriminator of pix2pix; no random input z but implemented drop out; use wgan;
             L1 distance as cycle consistency

DiscoGAN generator: conv, deconv and leaky relu; discriminator:conv+leaky relu  L2 distance as cycle consistency


\endinput
