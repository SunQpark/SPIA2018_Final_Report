\chapter{Final comments about polishing your report for publication}\label{Ch:Polishing}

Writing a good report is a serious challenge, requiring time and attention to details that are easily and often greatly underestimated by inexperienced writers.
So how does a good report acquire its final polish?

Your academic mentor will help guide your writing throughout your project.
He or she will be the first to review your report draft and edit it not only for style but also for technical correctness.
After you have satisfied your academic mentor with your draft, you will submit it to the RIPS program director for \emph{copy editing}, who will attend to matters of readability, grammar, and style.
After you submit your drafts for their review, it is likely they will return it to you with corrections, crossed out text, and possibly even suggestions
for overhauling whole parts of it. 
That is normal editing practice, and it is an expected part of the process of writing a professional-quality document.

Since your report is sponsored work, your sponsoring liaison should be given an opportunity to review it before its release.
But here's an important caution:  Don't submit a draft to your sponsor until 
\emph{after} it has been revised in compliance with suggestions from your academic mentor and the RIPS director.
It's good practice to give sponsors your most professional efforts---not your first drafts.
After you have satisfied the editing requirements of your academic mentor and the RIPS director, you should send your sponsor a \texttt{pdf} of a copy by email for review.
Your sponsor may suggest further changes.

\vspace{12pt}
\noindent You will facilitate the process of editing your report by submitting a single-sided printed copy for editing.
Double-sided is too hard to work with.
Although it is typical for copy editing to use double spacing of a manuscript to allow for editorial comments between lines, it is unnecessary for a RIPS report.
A table of {\em proofreader's marks} used by copy editors for mark-up, and used sometimes here at IPAM, is referenced in Appendix \ref{App:SourceLocation}.

After you have completed all the edits required by your academic mentor and the RIPS director, you can prepare final copies in two formats, respectively:
(1) a single-sided \texttt{pdf} as an electronic copy, which you can email to your sponsor, and 
(2) a slim double-sided copy for the final print version --- you can print this in the fatter single-sided format if your figures or text bleed through to the flip side of the page.
See Chapter \ref{Ch:ExtraAdvice} for a discussion of the special pagination requirements for double-sided copying.

Note that when you use \textsc{Adobe Reader} for printing your \texttt{pdf}, you are presented with options for {\em page scaling}.  You may have to play with this to get the  margins right.


\endinput
